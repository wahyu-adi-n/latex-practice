\documentclass{article}
\usepackage[table]{xcolor} % import library/package untuk warna pada table

% define my color
\definecolor{Gray}{gray}{0.99}
%\columncolor[RGB]{230, 242, 255}
%\columncolor[HTML]{AAACED}

\newcolumntype{a}{>{\columncolor{Gray}}c}
\newcolumntype{b}{>{\columncolor{green}}c}
\newcolumntype{m}{>{\columncolor{red}}c} % cara 1 memberi warna pada kolom 
\arrayrulecolor{black} % membuat border color menjadi biru

\begin{document}
    \begin{center}
        \begin{tabular}{| a | b | >{\columncolor{blue}}c |} %cara 2 memberi warna pada kolom
            \hline % horizontal line
            No. & Nama & NIM \\
            \hline 
            \rowcolor{blue}
            1   & \cellcolor{brown} Wahyu Adi Nugroho & A11.2019.12310 \\
            2   & Amanda Nur Cahyati & A11.2019.12311 \\
            3   & Ganjar Pranowo & A11.2019.12312 \\
            \hline
        \end{tabular}
    \end{center}

    \begin{center}
        \begin{tabular}{|c|c|c|c|}
            \hline
            & \multicolumn{3}{c|}{ Evaluation Metrics } \\ % colspan
            \cline{2-4}
            Algoritm & Precision & Recall & F1-Score \\
            \hline
            \rowcolor{red}
            CNN & 99.5 & 99.2 & 99.1 \\
            \hline            
        \end{tabular}
    \end{center}

\end{document}